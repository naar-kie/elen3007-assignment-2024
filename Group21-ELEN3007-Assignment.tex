\documentclass[a4paper,11pt]{article}
\usepackage{fullpage}
\usepackage{graphicx}
\usepackage{eqnarray,amsmath}


\pdfinfo{
/Title (REPORT TEMPLATE 2024)
/Author (Kgadile "Naar-Kie" Masemola)
/CreationDate (D:202303151636)
/ModDate (D:202408290530)
/Subject (ELEN3XXX Paper Format, 2024)
%/Keywords (ELEN3009, paper, project)
}

\title{ELEN3007A Group \underline{21} - Assignment 2024: \\ 
\large \emph{Application of Bayes’ Theorem for Locating a Robot’s
Position in an Enclosed Area}}
\author{Kgadile E Masemola (876729),  Thembinkosi Dhlamini (1234567),
 \\Siphokuhle Zulu (7654321), Lesego Gaborone (2176543)}
\date{September 13, 2024}

\begin{document}
\maketitle

\section*{Introduction and Background}

\section*{Assignment Answers}
\begin{enumerate}
  \item The given setup of the problem assumes that the photodetectors are placed on the x-axis above which the robot is located. Therefore, the signal comes from one side of the axis. This thus limits the range of the detectors to be within the range of $\pi$ (that is $-\frac{\pi}{2}$ to $\frac{\pi}{2}$). 
  \item Second
  \item Etc.
\end{enumerate}

\section*{Conclusion}
\end{document}